\documentclass[landscape,final,a0paper,fontscale=0.285]{baposter}

\usepackage{calc}
\usepackage{graphicx}
\usepackage{amsmath}
\usepackage{amssymb}
\usepackage{relsize}
\usepackage{multirow}
\usepackage{rotating}
\usepackage{bm}
\usepackage{url}

\usepackage{gensymb} 
\usepackage{graphicx}
\usepackage{multicol}
\usepackage[super]{nth}
\usepackage{amsmath}
\usepackage{subfig}
\usepackage{wrapfig}
\usepackage{float}
\usepackage[font=tiny,labelfont=bf]{caption} %scriptsize footnotesize
%\usepackage{times}
%\usepackage{helvet}
%\usepackage{bookman}
\usepackage{palatino}

\graphicspath{{images/}{../images/}}
\usetikzlibrary{calc}

\newcommand{\SET}[1]  {\ensuremath{\mathcal{#1}}}
\newcommand{\MAT}[1]  {\ensuremath{\boldsymbol{#1}}}
\newcommand{\VEC}[1]  {\ensuremath{\boldsymbol{#1}}}
\newcommand{\Video}{\SET{V}}
\newcommand{\video}{\VEC{f}}
\newcommand{\track}{x}
\newcommand{\Track}{\SET T}
\newcommand{\LMs}{\SET L}
\newcommand{\lm}{l}
\newcommand{\PosE}{\SET P}
\newcommand{\posE}{\VEC p}
\newcommand{\negE}{\VEC n}
\newcommand{\NegE}{\SET N}
\newcommand{\Occluded}{\SET O}
\newcommand{\occluded}{o}

\setlength{\columnsep}{1.5em}
\setlength{\columnseprule}{0mm}


\newcommand{\compresslist}{%
\setlength{\itemsep}{1pt}%
\setlength{\parskip}{0pt}%
\setlength{\parsep}{0pt}%
}
% ----------------------------------------------------------------------------
% *** START BIBLIOGRAPHY <<<
% ----------------------------------------------------------------------------
\usepackage[
	backend=biber, 
%	style = numeric,
	style=phys, % без doi
	maxbibnames=99,
	citestyle=numeric,
	giveninits=true,
	isbn=true,
	url=true,
	natbib=true,
	sorting=ndymdt,
	bibencoding=utf8,
	useprefix=false,
	language=auto, 
	autolang=other,
	backref=true,
	backrefstyle=none,
	indexing=cite,
]{biblatex}
\DeclareSortingTemplate{ndymdt}{
  \sort{
    \field{presort}
  }
  \sort[final]{
    \field{sortkey}
  }
  \sort{
    \field{sortname}
    \field{author}
    \field{editor}
    \field{translator}
    \field{sorttitle}
    \field{title}
  }
  \sort[direction=descending]{
    \field{sortyear}
    \field{year}
    \literal{9999}
  }
  \sort[direction=descending]{
    \field[padside=left,padwidth=2,padchar=0]{month}
    \literal{99}
  }
  \sort[direction=descending]{
    \field[padside=left,padwidth=2,padchar=0]{day}
    \literal{99}
  }
  \sort{
    \field{sorttitle}
  }
  \sort[direction=descending]{
    \field[padside=left,padwidth=4,padchar=0]{volume}
    \literal{9999}
  }
}

\addbibresource{Chris.bib}%   \tiny  \scriptsize \footnotesize \normalsize
\renewcommand*{\bibfont}{\scriptsize} % 

% ----------------------------------------------------------------------------
% *** END BIBLIOGRAPHY <<<
% ----------------------------------------------------------------------------

\begin{document}

%\definecolor{lightblue}{cmyk}{0.83,0.24,0,0.12}
\definecolor{lightblue}{rgb}{0.145,0.6666,1}

% Draw a video
\newlength{\FSZ}
\newcommand{\drawvideo}[3]{% [0 0.25 0.5 0.75 1 1.25 1.5]
   \noindent\pgfmathsetlength{\FSZ}{\linewidth/#2}
   \begin{tikzpicture}[outer sep=0pt,inner sep=0pt,x=\FSZ,y=\FSZ]
   \draw[color=lightblue!50!black] (0,0) node[outer sep=0pt,inner sep=0pt,text width=\linewidth,minimum height=0] (video) {\noindent#3};
   \path [fill=lightblue!50!black,line width=0pt] 
     (video.north west) rectangle ([yshift=\FSZ] video.north east) 
    \foreach \x in {1,2,...,#2} {
      {[rounded corners=0.6] ($(video.north west)+(-0.7,0.8)+(\x,0)$) rectangle +(0.4,-0.6)}
    }
;
   \path [fill=lightblue!50!black,line width=0pt] 
     ([yshift=-1\FSZ] video.south west) rectangle (video.south east) 
    \foreach \x in {1,2,...,#2} {
      {[rounded corners=0.6] ($(video.south west)+(-0.7,-0.2)+(\x,0)$) rectangle +(0.4,-0.6)}
    }
;
   \foreach \x in {1,...,#1} {
     \draw[color=lightblue!50!black] ([xshift=\x\linewidth/#1] video.north west) -- ([xshift=\x\linewidth/#1] video.south west);
   }
   \foreach \x in {0,#1} {
     \draw[color=lightblue!50!black] ([xshift=\x\linewidth/#1,yshift=1\FSZ] video.north west) -- ([xshift=\x\linewidth/#1,yshift=-1\FSZ] video.south west);
   }
   \end{tikzpicture}
}

\hyphenation{resolution occlusions}

%---------------------- POSTER STARTS ----------------------------------%

\begin{poster}%
  % Poster Options
  {
  % Show grid to help with alignment
  grid=false,
  % Column spacing
  colspacing=1em,
  % Color style
  bgColorOne=white,
  bgColorTwo=white,
  borderColor=lightblue,
  headerColorOne=black,
  headerColorTwo=lightblue,
  headerFontColor=white,
  boxColorOne=white,
  boxColorTwo=lightblue,
  % Format of textbox
  textborder=roundedleft,
  % Format of text header
  eyecatcher=true,
  headerborder=closed,
  headerheight=0.1\textheight,
%  textfont=\sc, An example of changing the text font
  headershape=roundedright,
  headershade=shadelr,
  headerfont=\Large\bf\textsc, %Sans Serif
  textfont={\setlength{\parindent}{1.5em}},
  boxshade=plain,
  %background=shade-tb,
  background=plain,
  linewidth=2pt
  }
  % Eye Catcher
  {\includegraphics[height=6em]{F6}} 
  % Title
  {\bf\textsc{Quality Assessment of Data from CHRIS/PROBA}\vspace{0.5em}}
  % Authors
  {\textsc{Polina Lemenkova}}
  % University logo
  {% The makebox allows the title to flow into the logo, this is a hack because of the L shaped logo.
    \includegraphics[height=3.5em]{images/logo}
  }


    \newcommand{\colouredcircle}{%
      \tikz{\useasboundingbox (-0.2em,-0.32em) rectangle(0.2em,0.32em); \draw[draw=black,fill=lightblue,line width=0.03em] (0,0) circle(0.18em);}}

%---------------------- COLUMN 1 ----------------------------------%

  \headerbox{Info}{name=info,column=0,row=0}{
\scriptsize{Student ID: 3 23369248. Course: 'GEOG6038 Calibration and Validation of Earth Observation Data'. Practical 1: Quality assessment of data from CHRIS/PROBA. Student: Lemenkova P. Supervisor: Prof. Dr. E. J. Milton. Funding: Erasmus Mundus MSc Scholarship GEM-L0022/2009/EW, University of Southampton, UK. 2009.

Acknowledgement: I thank Brian Amberg and Thomas Vetter (Univerait\"{a}t Basel) for baposter \LaTeX \space source code used for this poster.
 }}
 
\headerbox{Introduction}{name=introduction,column=0,below=info}{%,bottomaligned=speed

  \noindent{\centering\includegraphics[width=0.80\linewidth]{F6}\\}
  \indent \footnotesize{Image: Vertical air photo of Thorney Island.}
\footnotesize{  
  \indent ROI: Thorney Island, Chichester harbour (UK): unique wetland environment, a place for rare bird colonies. Monitoring this place is important for environmental management. CHRIS/PROBA image characteristics: 18 bands, 07/10/2004, 17m ground resolution. 
  }}
  
\headerbox{Data: CHRIS/PROBA Image}{name=data,column=0,span=1,above=bottom}{
\footnotesize{
CHRIS (\emph{Compact High Resolution Imaging Spectrometer}): new imaging spectrometer carried on board a space platform PROBA, http://www.chris-proba.org.uk/. More information: Cutter M.A., Lobb D.R. Cockshott R. (2000) Compact High Resolution Imaging Spectrometer. Elsevier Science Ltd, Kent, UK. Kuusk A., Kuusk J., Lang M.. A dataset for the validation of reflectance models. Remote Sensing of Environment 113 (2009) 889–892 }

\footnotesize{PROBA: \emph{Project for On Board Autonomy}. The satellite was successfully launched in late October 2001, Shriharikota (India). Info: Bermin J. PROBA – Project for On-Board Autonomy and web: http://earth.esa.int/missions/thirdpartymission/proba.html  
   \vspace{0.3em}
  }}

%---------------------- COLUMN 2 ----------------------------------%

\headerbox{CHRIS Image Quality}{name=quality,column=1,row=0}{%,bottomaligned=problem
\footnotesize{For the quality of CHRIS images .hdr files were examined:}
\begin{enumerate}\compresslist
   \item CHRIS...47AO\_41.hdr (taken at +36\degree)
   \item CHRIS...47A1\_41.hdr (taken at -36\degree)
   \item CHRIS...47A2\_41.hdr (taken at +55\degree)
   \item CHRIS...47A3\_41.hdr (taken at -55\degree)
   \item CHRIS...479F\_41.hdr (taken at nadir)
\end{enumerate}

\footnotesize{Images taken at the nadir are of good quality, while those at different angles have defects}

%\noindent\begin{tabular}{r@{\hspace{0.3em}}c@{\hspace{1.5em}}c@{}}
%\begin{sideways}{\makebox[0.37\linewidth][c]{CHRIS .hdr files}}\end{sideways} &
%	\includegraphics[width=0.42\linewidth]{F9} & \includegraphics[width=0.40\linewidth]{F10} \\[2em]
%\end{tabular}
    \drawvideo{2}{40}{%
	\includegraphics[width=0.5\linewidth]{F9}%
	\includegraphics[width=0.48\linewidth]{F10}%
    }
\footnotesize{\textcolor{red}{Left}: Comparing images taken at +55\degree dgr (47A2\_41) and nadir images (479F\_41) right
\textcolor{red}{Right}: Images taken at +36\degree dgr (47A0\_41), left and nadir images (479F\_41) right.}

\noindent\begin{tabular}{r@{\hspace{0.3em}}c@{\hspace{1.5em}}c@{}}
\begin{sideways}{\makebox[0.37\linewidth][c]{CHRIS .hdr files}}\end{sideways} &
	\includegraphics[width=0.36\linewidth]{F3} & \includegraphics[width=0.48\linewidth]{F11}
\end{tabular}

\footnotesize{
\textcolor{red}{Left}: Inverted Image received from bands Combination 4(R)-2(G)-1(B). 

\textcolor{red}{Right}: Images taken at +36\degree and -36\degree (CHRIS 47A0\_41 and CHRIS 47A1\_41) both have inverted direction.}

  }

\headerbox{Spectral Bands}{name=background model,column=1,below=quality}{%,bottomaligned=speed

\noindent\begin{tabular}{r@{\hspace{0.2em}}c@{\hspace{1.5em}}c@{}}
\begin{sideways}{\makebox[0.3\linewidth][c]{\scriptsize{Narrow bands}}}\end{sideways} &
	\includegraphics[width=0.40\linewidth]{F4} &
	\includegraphics[width=0.40\linewidth]{F5}
\end{tabular}

\noindent\begin{tabular}{r@{\hspace{0.2em}}c@{\hspace{1.5em}}c@{}}
\begin{sideways}{\makebox[0.3\linewidth][c]{\scriptsize{Narrow bands}}}\end{sideways} &
	\includegraphics[width=0.20\linewidth]{F8} &
	\includegraphics[width=0.45\linewidth]{F12}
\end{tabular}
}
  
%---------------------- COLUMN 3 ----------------------------------%

\headerbox{Image Noise}{name=noise,column=2,span=2,row=0}{

    \drawvideo{5}{40}{%
	\includegraphics[width=0.5\linewidth]{F1}%
	\includegraphics[width=0.42\linewidth]{F2}%
    }
       \vspace{-2.0em}
\scriptsize{
       \begin{multicols}{2}
2 types of noise affecting CHRIS images:
\begin{enumerate}\compresslist
   \item vertical noise (vertical stripes corrected by comparing values of neighbor pixels)
   \item horizontal noise (easy to detect and correct using horizontal profile of each file
\end{enumerate}
Example of vertical noises in CHRIS image (left). Image source: Garcia J.G., Moreno J. Removal of noises in CHRIS/PROBA images: Application to the Sparc Campaign Data. Correction of noises can be made through DIELMO 3D Methodology. Example of horizontal noise in CHRIS image.
    \end{multicols}
}
      \vspace{-0.6em}
      
   \drawvideo{5}{40}{%
        \includegraphics[width=0.197\linewidth]{F13}%
          \includegraphics[width=0.194\linewidth]{F14}%
          \includegraphics[width=0.199\linewidth]{F17}%
          \includegraphics[width=0.21\linewidth]{F18}%
          \includegraphics[width=0.197\linewidth]{F19}%
      }
\scriptsize{
  \begin{tabular*}{\linewidth}{*{5}{@{}p{0.2\linewidth}@{}}}
    {\hfill{}CHRIS\_CH\_041007\_479F\_41 Bands\_12-8-1\hfill{}}  &
    {\hfill{}CHRIS\_CH\_041007\_479F\_41 Bands\_7-4-1\hfill{}} &
    {\hfill{}CHRIS\_CH\_041007\_479F\_41 Bands\_11-4-2\hfill{}} &
    {\hfill{}CHRIS\_CH\_041007\_479F\_41 Bands\_6-5-2\hfill{}} &
    {\hfill{}CHRIS\_CH\_041007\_479F\_41 Bands\_15-4-1\hfill{}} 
  \end{tabular*}
}
}

%----------------------

\headerbox{Image Inversion}{name=speed,column=2,span=1,row=0,below=noise,aligned=bib}{
\scriptsize{
Inverted Image, bands: \textcolor{red}{Left}:13-12-11. \textcolor{red}{Right}: 9(R)-10(G)-11(B).
}

\drawvideo{2}{20}{%
	\includegraphics[width=0.4\linewidth]{F15}%
	\includegraphics[width=0.44\linewidth]{F16}%
}
%\indent 
\scriptsize{
$\hookleftarrow$ \textcolor{red}{Left}: CHRIS Superspectral Land: Many narrow bands around the red-edge. Image source: Barnsley M.J. et al. The PROBA/CHRIS Mission: A Low-Cost Smallsat for Hyperspectral, Multi-Angle, Observations of the Earth Surface and Atmosphere. .
$\Leftarrow$\textcolor{red}{Right}: CHRIS Superspectral Water: many narrow bands in visible wavelengths. Figure contrasts per frame evidence for each patch with or w/o a background model.
$\hookleftarrow$Quality natural-colored image of wetlands: nadir-taken CHRIS image with bands combination of corresponding spectral channels. View angles in a CHRIS/PROBA acquisition. 
}
}
 
\headerbox{Bibliography}{name=bib,column=3,above=bottom}{
\scriptsize{Author's publications on Geography, Remote Sensing and GIS}:\\
    \tiny
\vspace{-2.5em}
	\nocite{*}
\printbibliography[heading=none]
}

\end{poster}

\end{document}
